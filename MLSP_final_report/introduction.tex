\section{Introduction}
\label{sec:intro}
Voice conversion technology can have applications ranging from entertainment to speaker recognition. It would be beneficial to the media industry to reproduce the voices of famous actors or actresses after they have passed. It would also be useful in translating media to different languages in the original actress or actor's voice. It may also help in the grieving process—--how many people wish they could hear a loved one’s voice one last time after they pass? Voice conversion research can also contribute to fundamental research in other areas--for instance, achieving voice conversion means voice identification must be possible.

A fair amount of work has been done on this topic. For example, some methods propose to learn source--target relationship from a number utterances \cite{stylianou2009voice}, some methods are parametric, which learns a mapping function between a source to target in some feature spaces\cite{stylianou1998continuous}\cite{kawahara1997speech}.An alternative approach is called unit selection\cite{duxans2006voice}\cite{jin2016cute}. The idea is to select the segments from training sets of a target voice, which correspond to the speech content of source voice, then concatenating the seg- ments with smooth transition.

Our contribution is an implementation of the voice conversion work-flow algorithm given by Stylianou et al \cite{stylianou1998continuous}. In short this involves feature extraction of the fundamental frequency $f_0$ and mel's cepstrum $mcep$, dynamic time wrapping (DTW) to align the speech of our target and source speaker in the training data, conversion to the target voice using VQ and Full Conversion algorithms with Gaussian Mixture Models, $F_0$ transformation to match the fundamental frequency to the target speaker, and reproduction of an audio signal. We wrap up our report by evaluating of distortion as a measure of performance. 
